\documentclass[a4paper, twoside, 12pt]{report}
\usepackage{graphicx}


% font?
\title{Durak with Artificial Intelligence}
\author{Seyhan Van Khan}
\date{June 2024}
\makeatletter

\begin{document}

\bibliographystyle{vancouver}

\begin{titlepage}
	\centering
	{\huge \@title}
	\includegraphics[width=5cm]{./imperial.png}
	\vfill

\end{titlepage}

\section{Notes}

A K Q J 10 9 8 7 6 \\
A K Q J 10 9 8 7 6 \\
A K Q J 10 9 8 7 6 \\
A K Q J 10 9 8 7 6 (Trumps) \\
\\
8 unknown trumps, 1 known \\

Talon: 24 cards (with 1 known trump) \\

interesting thought - if we knew all the cards positions and opponents hands --- it would be a solved game right?

\subsection*{Strategies}
It's good to have a vertical collection (consecutive cards of same suit). It means u can attack the entire set one by one, without them replying. Controlling a suit forces opponent to lose trumps or take the cards. When attacking a set, start with the lowest, but skip one card in case you get attacked with them later on. That one card stops their momentum, and brings the tempo back to you.




\end{document}